%%%%%%%%%%%%%%%%%%%%%%%%%%%%%%%%%%%%%%%%%%%%%%%%%%%%%%%%%%%%%%%%%%%%%%%%%%%%%%%
%%%%%%%%%%%%%%%%%%%%%%%%                                %%%%%%%%%%%%%%%%%%%%%%%
%%%%%%%%%%%%%%%%%%%%%%%%                                %%%%%%%%%%%%%%%%%%%%%%%
%%%%%%%%%%%%%%%%%%%%%%%% LUALATEX-MEMOIR PREAMBLE FILE  %%%%%%%%%%%%%%%%%%%%%%%
%%%%%%%%%%%%%%%%%%%%%%%% ALL SECTIONS ARE MARKED        %%%%%%%%%%%%%%%%%%%%%%%
%%%%%%%%%%%%%%%%%%%%%%%%                                %%%%%%%%%%%%%%%%%%%%%%%
%%%%%%%%%%%%%%%%%%%%%%%%                                %%%%%%%%%%%%%%%%%%%%%%%
%%%%%%%%%%%%%%%%%%%%%%%%%%%%%%%%%%%%%%%%%%%%%%%%%%%%%%%%%%%%%%%%%%%%%%%%%%%%%%%
% TeXworks directives
% !TeX root = main.tex
% !TeX program = LuaLaTeX
% !TeX encoding = utf-8
%%%%%%%%%%%%%%%%%%%%%%%% CALLING NEEDED PACKAGES   %%%%%%%%%%%%%%%%%%%%%%%%%%%%
\usepackage[dutch,english]{babel}

\usepackage[svgnames]{xcolor}
\definecolor{White}{rgb}{1,1,.99}%very strange. otherwise white is not ok


\usepackage{graphicx,luatextra}
\makeatletter
\let\c@lotdepth\relax
\let\c@lofdepth\relax
\makeatother
\DisemulatePackage{subfigure}
%\usepackage{flafter}% uncomment to place figures after the referencing
\usepackage[normalsize,tight,center,hang]{subfigure}
\usepackage{multirow,rotating,colortbl,tabu,afterpage,pdflscape}

\usepackage{etoolbox}% adds some programming commands

\usepackage[perpage,hang,marginal,bottom,symbol]{footmisc}

\usepackage[version=3]{mhchem}
\mhchemoptions{textfontcommand=\normalfont}
\mhchemoptions{mathfontcommand=\normalfont}

\usepackage{siunitx}

\usepackage{amsmath}
\usepackage{minted}
\usepackage[math-style=ISO,bold-style=ISO]{unicode-math}
\PassOptionsToPackage{fontspec}{silent}

%\usepackage{xpatch}% NEEDED BY BIBLATEX
\usepackage[autostyle=true]{csquotes}
\usepackage[bibstyle=ieee,citestyle=numeric-comp,%
  backend=biber]{biblatex}

\newcommand*{\refcolor}{DodgerBlue}
\definecolor{specialblue}{RGB}{0,85,212}
%\definecolor{specialblue}{RGB}{0,0,0}%comment to produce color links
%\renewcommand*{\refcolor}{Black}%comment to produce color links
\newcommand\MYhyperrefoptions{bookmarks=true,bookmarksnumbered=true,%
bookmarksopen=true,pdfnewwindow=true,pdfpagelayout=OneColumn,%
pdfpagemode={UseOutlines},plainpages=false,pdfpagelabels=true,%
breaklinks=true,unicode=true,linktocpage=true,%ocgcolorlinks,%
colorlinks=true,linkcolor=\refcolor,citecolor=\refcolor,%
urlcolor=\refcolor,filecolor=\refcolor,hyperfootnotes=false,%
plainpages=false,pdftitle={Development of a Stretchable Platform for the 
Fabrication of Biocompatible Microsystems},pdfsubject={PhD Dissertation},%
pdfauthor={Amir Jahanshahi},pdfcreator={LuaLaTeX memoir document class},%
pdfkeywords={Microvalves,Micropumps,Microfluidics,Soft lithography,Piezo Pump,%
Stretchable}}


\usepackage{varioref}%extends \pageref via \vref to automate "on the next page"
\usepackage[\MYhyperrefoptions]{hyperref}%
\usepackage{cleveref}%the order of varioref and cleveref is import
%more info: http://tex.stackexchange.com/questions/83037/difference-between-ref
%-varioref-and-cleveref-decision-for-a-thesis
\usepackage{xr}   % for cross references between multiple chapters

%for drop shadow effect
\usepackage{tikz}
\usetikzlibrary{shadows,calc}

%for double column text
\usepackage{adjmulticol}

%better control over lists
\usepackage{enumitem}

\usepackage[activate={true,nocompatibility},%
            %kerning=true, %does not work yet with lua
            tracking=true,
            %spacing=true, %does not work yet with lua
	          factor=1100,final]{microtype}
% activate={true,nocompatibility} - activate protrusion and expansion
% final - enable microtype; use "draft" to disable
% tracking=true, kerning=true, spacing=true - activate these techniques
% factor=1100 - add 10% to the protrusion amount (default is 1000)
% stretch=10, shrink=10 - reduce stretchability/shrinkability (default is 20/20)

\usepackage{pdfpages}
%%%%%%%%%%%%%%%%%%%%%%%%%%%%%%%%%%%%%%%%%%%%%%%%%%%%%%%%%%%%%%%%%%%%%%%%%%%%%%%
%%%%%%%%%%%%%%%%%%%%%%%%%%%%%%%%%%%%%%%%%%%%%%%%%%%%%%%%%%%%%%%%%%%%%%%%%%%%%%%



%%%%%%%%%%%%%%%%%%%%%%%%%%%%%%%%%%%%%%%%%%%%%%%%%%%%%%%%%%%%%%%%%%%%%%%%%%%%%%%
%%%%%%%%%%%%%%%%%%%%%%%% SELECT ALL FONTS AND MACRO FONTS %%%%%%%%%%%%%%%%%%%%%
%%%%%%%%%%%%%%%%%%%%%%%% OPTIMUM FONT SIZE CALCULATION %%%%%%%%%%%%%%%%%%%%%%%%
\defaultfontfeatures{Scale=MatchLowercase,Ligatures={TeX}}

\setmainfont{Minion Pro}
\setsansfont[Numbers={Proportional,OldStyle}]{Latin Modern Sans}
\setmonofont{Monaco}
\setmathfont{Cambria Math}
\setmathfont[range=\mathup]  {Minion Pro}
\setmathfont[range=\mathbfup]{Minion Pro Bold}
\setmathfont[range=\mathbfit]{Minion Pro Bold Italic}
\setmathfont[range=\mathit]  {Minion Pro Italic}

\newfontfamily\tofont{Minion Pro}
\newfontfamily\thanksfont{Minion Pro}
\newfontfamily\summaryfont{Minion Pro}
\newfontfamily\epifont{Warnock Pro Light}

\newfontfamily\captionfont{Minion Pro Med Cond}
\newfontfamily\captionfonttitle{Myriad Pro}

\newfontfamily\headingfont[Color=specialblue]{Myriad Pro Cond}

\newfontfamily\chapternumfont[Color=specialblue]{Minion Pro}
\newfontfamily\chaptertitlefont{TexGyreHeros}

\newfontfamily\pagenumfont[Scale=1.4]{UniversLTStd-LightUltraCn.otf}

\newfontfamily\headerfont[Scale=1.1,WordSpace=1.2,Color=Black]{%
                          UniversLTStd-LightUltraCn.otf}
\newcommand*{\chapteroverviewheadingfont}{%
\headingfont\addfontfeatures{Color=specialblue,WordSpace={1.6}}}
%%%%%%%%%%%%%%%%%%%%%%%%%%%%%%%%%%%%%%%%%%%%%%%%%%%%%%%%%%%%%%%%%%%%%%%%%%%%%%%
%%%%%%%%%%%%%%%%%%%%%%%%%%%%%%%%%%%%%%%%%%%%%%%%%%%%%%%%%%%%%%%%%%%%%%%%%%%%%%%



%%%%%%%%%%%%%%%%%%%%%%%%%%%%%%%%%%%%%%%%%%%%%%%%%%%%%%%%%%%%%%%%%%%%%%%%%%%%%%%
%%%%%%%%%%%%%%%%%%%%%%%% PAGE LAYOUT DEFITION - MEMOIR  %%%%%%%%%%%%%%%%%%%%%%%
\setstocksize{297mm}{210mm}
\settrimmedsize{240mm}{160mm}{*}
\settrims{28mm}{25mm}% define where to put the page in stock
%\settrims{0pt}{0pt}
\settypeblocksize{186.66mm}{106.66mm}{*}
%\settypeblocksize{*}{\lxvchars}{1.618}
\setlrmargins{*}{*}{1.618}
\setulmargins{*}{*}{1.618}
\setheadfoot{1.5\onelineskip}{3\onelineskip}% <headheight><footskip>
\setheaderspaces{*}{*}{1.618}
%<marginparsep><marginparwidth> is set below
\setmarginnotes{0.04\textwidth}{0.8\spinemargin}{2\onelineskip}
\checkandfixthelayout
%\showtrimsoff%uncomment to remove the trim lines
\openright%open chapter on recto page always
\newlength{\totalmarginwidth}%=\marginparwidth+\marginparsep
\setlength{\totalmarginwidth}{\marginparwidth}
\addtolength{\totalmarginwidth}{\marginparsep}
\newlength{\totaltextwidth}%=\textwidth+\totalmarginwidth
\setlength{\totaltextwidth}{\totalmarginwidth}
\addtolength{\totaltextwidth}{\textwidth}
%%%%%%%%%%%%%%%%%%%%%%%%%%%%%%%%%%%%%%%%%%%%%%%%%%%%%%%%%%%%%%%%%%%%%%%%%%%%%%%
%%%%%%%%%%%%%%%%%%%%%%%%%%%%%%%%%%%%%%%%%%%%%%%%%%%%%%%%%%%%%%%%%%%%%%%%%%%%%%%



%%%%%%%%%%%%%%%%%%%%%%%%%%%%%%%%%%%%%%%%%%%%%%%%%%%%%%%%%%%%%%%%%%%%%%%%%%%%%%%
%%%%%%%%%%%%%%%%%%%%%%%% MAKING HEADERS AND FOOTERS %%%%%%%%%%%%%%%%%%%%%%%%%%%
\makeatletter
\makepagestyle{genomics}
\setlength{\headwidth}{\totaltextwidth}
\makerunningwidth{genomics}{\headwidth}
\makeheadposition{genomics}{flushright}{flushleft}{}{}
\makepsmarks{genomics}{%
  %\nouppercaseheads
  \createmark{chapter}{both}{shownumber}{\@chapapp\thinspace}{:\space}
  \createmark{section}{right}{shownumber}{}{.\space}
  \createplainmark{toc}{both}{\contentsname}
  \createplainmark{lof}{both}{\listfigurename}
  \createplainmark{lot}{both}{\listtablename}
  \createplainmark{bib}{both}{\bibname}
  \createplainmark{index}{both}{\indexname}
  \createplainmark{glossary}{both}{\glossaryname}
}
\makeevenhead{genomics}{%
  \pagenumfont{\color{specialblue}{\thepage}}\quad\headerfont\bfseries%
\leftmark}{}{}
\makeoddhead{genomics}{}{}{%
\headerfont\bfseries\rightmark\pagenumfont\quad%
{\color{specialblue}\thepage}}
\pagestyle{genomics}
\makeatother
%%%%%%%%%%%%%%%%%%%%%%%%%%%%%%%%%%%%%%%%%%%%%%%%%%%%%%%%%%%%%%%%%%%%%%%%%%%%%%%
%%%%%%%%%%%%%%%%%%%%%%%%%%%%%%%%%%%%%%%%%%%%%%%%%%%%%%%%%%%%%%%%%%%%%%%%%%%%%%%



%%%%%%%%%%%%%%%%%%%%%%%%%%%%%%%%%%%%%%%%%%%%%%%%%%%%%%%%%%%%%%%%%%%%%%%%%%%%%%%
%%%%%%%%%%%%%%%%%%%%%%%% CHAPTER PRINTING STYLE %%%%%%%%%%%%%%%%%%%%%%%%%%%%%%%
%This part prints the the big chapter number on each chapter
\newlength{\chapternumwidth}
\newlength{\chapternumheight}
\newcommand{\chapternumtext}{\thechapter}
\newcommand{\chapternumbox}[1]{%
\chapternumfont\bfseries%
\addfontfeature{Color=specialblue,Scale=#1}%
\settowidth{\chapternumwidth}{\thechapter}%
\settoheight{\chapternumheight}{\thechapter}%
\renewcommand{\chapternumtext}{\makebox[1.3\chapternumwidth]{\thechapter}}%
\settowidth{\chapternumwidth}{\chapternumtext}%
\settoheight{\chapternumheight}{\chapternumtext}%
\resizebox{\chapternumwidth}{1.5\chapternumheight}{%
\chapternumtext}}%

\newcommand{\drawchapnum}{%
\ifanappendix% If appendix print in the middle otherwise right.
  \makebox[\textwidth][c]{\chapternumbox{4}}
\else
  \makebox[\totaltextwidth][r]{\chapternumbox{7}}
\fi} %%argument is scale

%This part prints the chapter title
\newsavebox{\chaptitlebox}
\newlength{\chaptitlewidth}
\newlength{\chaptitleheight}
\newenvironment{chaptitle}{%
  \begin{lrbox}{\chaptitlebox}
  \begin{minipage}[b]{\textwidth}\raggedright}{%
  \end{minipage}\end{lrbox}%
  \settowidth{\chaptitlewidth}{\usebox{\chaptitlebox}}%
  \settoheight{\chaptitleheight}{\usebox{\chaptitlebox}}%
  \resizebox{\chaptitlewidth}{\chaptitleheight}{%
    \usebox{\chaptitlebox}}}

\newcommand{\printchaptitle}[1]{%
  \chaptitlefont\begin{chaptitle}\DoubleSpacing#1\end{chaptitle}}

\newif\ifnumberedchap
\numberedchaptrue
\makechapterstyle{me-pedersen}{
\setlength\beforechapskip{0pt}%\baselineskip is already there
\renewcommand*\chaptitlefont{\chaptertitlefont\addfontfeature{Scale=1.8}}
%\renewcommand*\chapternamenum{}
\renewcommand\printchaptername{}%Prints Chapter for example
\renewcommand\printchapternum{\drawchapnum}%prints chapter number
\setlength{\midchapskip}{\baselineskip}
\renewcommand\printchapternonum{\global\numberedchapfalse}
\renewcommand\afterchapternum{\null\par\vskip\midchapskip}
\renewcommand\printchaptertitle[1]{%
  \ifnumberedchap% checks whether chapter or chapter* is ordered.
  {\color{specialblue}\hrule height 2pt}%
  \vskip\onelineskip%
  \fi%
  \raisebox{0pt}{%
  \printchaptitle{##1}\par}}
\renewcommand{\afterchaptertitle}{%
  \vskip\onelineskip
  \ifnumberedchap% checks whether chapter or chapter* is ordered.
  {\color{specialblue}\hrule height 2pt}%
  \fi%
  \global\numberedchaptrue%
  \vskip 0.5\afterchapskip}%
}

\chapterstyle{me-pedersen}
\newcommand*{\sectionheadstyle}[1]{\bfseries\headingfont%
  \addfontfeature{Scale=1.55}\parbox{0.6\textwidth}{\raggedright\OnehalfSpacing#1}}
\setsecindent{-1em}%This should be equal to parindent
\setsecheadstyle{\sectionheadstyle}
\newcommand{\subsectionheadstyle}[1]{\headingfont\bfseries%
  \addfontfeature{Scale=1.2,Color=black}\parbox{0.6\textwidth}{\raggedright#1}}
\setsubsecindent{-1em}%This should be equal to parindent
\setsubsecheadstyle{\subsectionheadstyle}
\newcommand{\subsubsectionheadstyle}[1]{\headingfont\itshape%
  \addfontfeature{Scale=1.0,Color=black} #1}
\setsubsubsecheadstyle{\subsubsectionheadstyle}
% We add an optional suffix to numbers when printing the section header
%http://tex.stackexchange.com/questions/117611/change-only-display-of-thesubsection-not-the-actual-value/117623?noredirect=1#117623 
\setsecnumformat{\csname prefix@#1\endcsname
  \csname the#1\endcsname
  \csname suffix@#1\endcsname\quad}
\renewcommand*{\thesubsubsection}{\textup{\Alph{subsubsection}}}
% Access to the internals
\makeatletter
% redefine the prefix to the subsection number when \ref is used
\renewcommand{\p@subsubsection}{\thesubsection.}
% only \suffix@subsection needs a definition
\newcommand{\prefix@subsubsection}{\begingroup\Large}
\newcommand{\suffix@subsubsection}{.\endgroup}
\makeatother
\setsecnumdepth{subsubsection}

%Chapter Overview environment
%http://tex.stackexchange.com/questions/114378/flush-bottom-an-environment/114752?noredirect=1#comment253671_114752
\newenvironment{chapteroverview}{\par\vspace*{\fill}%
    \noindent\begin{minipage}{\textwidth}%
    \begin{center}\chapteroverviewheadingfont% 
    \LARGE OUTLINE OF THIS CHAPTER
    \end{center}%
    \vspace*{-6pt}\large\sffamily %
    %\hrule height 0.5pt width \textwidth \vspace*{10pt}%
    }%End of the begining commands
  {\vspace*{10pt}%
  %\hrule height 0.5pt width \textwidth
  \end{minipage}\newpage %end this page
  }
%%%%%%%%%%%%%%%%%%%%%%%%%%%%%%%%%%%%%%%%%%%%%%%%%%%%%%%%%%%%%%%%%%%%%%%%%%%%%%%
%%%%%%%%%%%%%%%%%%%%%%%%%%%%%%%%%%%%%%%%%%%%%%%%%%%%%%%%%%%%%%%%%%%%%%%%%%%%%%%



%%%%%%%%%%%%%%%%%%%%%%%%%%%%%%%%%%%%%%%%%%%%%%%%%%%%%%%%%%%%%%%%%%%%%%%%%%%%%%%
%%%%%%%%%%%%%%%%%%%%%%%%%%% TOC, LOT, LOF %%%%%%%%%%%%%%%%%%%%%%%%%%%%%%%%%%%%%
%source: http://hstuart.dk/2007/05/26/styling-the-table-of-contents/
\renewcommand{\cftchapterleader}{\hspace{\marginparsep}}
\renewcommand{\cftchapterafterpnum}{\cftparfillskip}

\renewcommand{\cftsectionleader}{\hspace{\marginparsep}}
\renewcommand{\cftsectionafterpnum}{\cftparfillskip}

\setcounter{tocdepth}{2}
\renewcommand{\cftsubsectionleader}{\hspace{\marginparsep}}
\renewcommand{\cftsubsectionafterpnum}{\cftparfillskip}
%%%%%%%%%%%%%% FROM HERE ADDED BY ME %%%%%%%%%%%%%%%%%%%%%%%%%%%%%%%%%%%%%%%%%%
\setrmarg{2em plus 1fil}

\renewcommand{\cftfigureleader}{\hspace{\marginparsep}}%page after title
\renewcommand{\cftfigureafterpnum}{\cftparfillskip}%fill line after num

\renewcommand{\cfttableleader}{\hspace{\marginparsep}}
\renewcommand{\cfttableafterpnum}{\cftparfillskip}

\cftsetindents{subsection}{4.5em}{3.9em}%indent the subsection names
\addto\captionsenglish{\renewcommand*{\contentsname}{Table of Contents}}
\addto\captionsenglish{\renewcommand*{\listfigurename}{List of Selected Figures}}
\addto\captionsenglish{\renewcommand*{\listtablename}{List of Selected Tables}}
\renewcommand*{\cftchapterfont}{\sffamily\large\bfseries}%self explanatory
\renewcommand*{\cftsectionfont}{\sffamily}%self explanatory
\renewcommand*{\cftsubsectionfont}{\sffamily}%self explanatory
%%%%%%%%%%%%%%%%%%%%%%%%%%%%%%%%%%%%%%%%%%%%%%%%%%%%%%%%%%%%%%%%%%%%%%%%%%%%%%%
%%%%%%%%%%%%%%%%%%%%%%%%%%%%%%%%%%%%%%%%%%%%%%%%%%%%%%%%%%%%%%%%%%%%%%%%%%%%%%%



%%%%%%%%%%%%%%%%%%%%%%%%%%%%%%%%%%%%%%%%%%%%%%%%%%%%%%%%%%%%%%%%%%%%%%%%%%%%%%%
%%%%%%%%%%%%%%%%%%%%%%%%%%% GLOASSARIES %%%%%%%%%%%%%%%%%%%%%%%%%%%%%%%%%%%%%%%
\usepackage[xindy,nonumberlist]{glossaries}% To avoid the annoying glossary 
% text after the section heading:
%http://tex.stackexchange.com/questions/28659/pagestyleruled-adds-superfluous-text-in-glossary
%\usepackage{bidi}% SHOULD BE THE LAST PACKAGE TO LOAD - Not working in Lua
\loadglsentries{main-acro}
\renewcommand*{\glspostdescription}{} % remove the annoying dot after entry
\renewcommand*{\glossaryname}{Glossary}% will appear in TOC
\makeglossaries
\glsaddall    % add all the glossary regardless of them being used.
\glossarystyle{long} 
\renewcommand*{\glstextformat}[1]{\textcolor{black}{#1}}
%%%%%%%%%%%%%%%%%%%%%%%%%%%%%%%%%%%%%%%%%%%%%%%%%%%%%%%%%%%%%%%%%%%%%%%%%%%%%%%
%%%%%%%%%%%%%%%%%%%%%%%%%%%%%%%%%%%%%%%%%%%%%%%%%%%%%%%%%%%%%%%%%%%%%%%%%%%%%%%



%%%%%%%%%%%%%%%%%%%%%%%%%%%%%%%%%%%%%%%%%%%%%%%%%%%%%%%%%%%%%%%%%%%%%%%%%%%%%%%
%%%%%%%%%%%%%%%%%%%%%%%% MISC %%%%%%%%%%%%%%%%%%%%%%%%%%%%%%%%%%%%%%%%%%%%%%%%%
\setlength{\epigraphwidth}{0.6\textwidth}
\midsloppy% sth between \fussy and \sloppy
\OnehalfSpacing
\setlength{\parskip}{0pt}%remove inter-paragrph spacing - looks very bad
\setlength{\parindent}{1.0em}%indentation at the start of the new paragraphs
\sisetup{
range-phrase = --,
separate-uncertainty=true,
mode=math,%All units and numbers are set in text mode
detect-all,%change font appropriately according to context
}
\AtBeginEnvironment{minted}{\small\SingleSpacing}% Use small font for codes
\newcommand{\inputcode}[2]{\inputminted[mathescape,%
                                        linenos=true,%
                                        formatcom=\small\SingleSpacing]{#1}{#2}}%
\renewcommand{\theFancyVerbLine}{\sffamily\scriptsize
\textcolor[rgb]{0.5,0.5,1.0}{\oldstylenums{\arabic{FancyVerbLine}}}}
\newcommand{\twoblankpages}{\pagenumbering{gobble}\mbox{}\thispagestyle{empty}%
                            \clearpage\pagenumbering{gobble}\mbox{}%
                            \thispagestyle{empty}\clearpage}
%%%%%%%%%%%%%%%%%%%%%%%%%%%%%%%%%%%%%%%%%%%%%%%%%%%%%%%%%%%%%%%%%%%%%%%%%%%%%%%
%%%%%%%%%%%%%%%%%%%%%%%%%%%%%%%%%%%%%%%%%%%%%%%%%%%%%%%%%%%%%%%%%%%%%%%%%%%%%%%



%%%%%%%%%%%%%%%%%%%%%%%%%%%%%%%%%%%%%%%%%%%%%%%%%%%%%%%%%%%%%%%%%%%%%%%%%%%%%%%
%%%%%%%%%%%%%%%%%%%%%%%% FLOAT OPTIONS                  %%%%%%%%%%%%%%%%%%%%%%%
\graphicspath{{Pictures/}}

\crefname{table}{table}{tables}
\Crefname{table}{Table}{Tables}
\crefname{figure}{Fig.}{fig.}
\Crefname{figure}{Fig.}{Fig.}
\crefname{section}{Sec.}{Sec.}
\Crefname{section}{Sec.}{Sec.}
\crefname{equation}{Equ.}{Equ.}
\Crefname{equation}{Equ.}{Equ.}

\setfloatlocations{figure}{t}
\setfloatlocations{table}{t}
\setcounter{topnumber}{3}
\setcounter{bottomnumber}{0}
\renewcommand{\topfraction}{0.8}
\renewcommand{\bottomfraction}{0.6}
\renewcommand{\textfraction}{0.2}
\renewcommand{\floatpagefraction}{0.8}
\renewcommand{\textfloatsep}{10pt} % this is the default
%%%%%%%%%%% CREATING DROP SHADOW EFFECT %%%%%%%%%%%%%%%%%%%%%%%%%%%%%%%%%%%%%%%
% adopted from here: http://tex.stackexchange.com/questions/81842/creating-a-
%drop-shadow-with-guassian-blur#comment174996_81842
% some parameters for customization
\def\shadowshift{1.5pt,-1.5pt}
\def\shadowradius{3pt}

\colorlet{innercolor}{black!60}
\colorlet{outercolor}{gray!05}

% this draws a shadow under a rectangle node
\newcommand\drawshadow[1]{
    \begin{pgfonlayer}{shadow}
        \shade[outercolor,inner color=innercolor,outer color=outercolor] ($(#1.south west)+(\shadowshift)+(\shadowradius/2,\shadowradius/2)$) circle (\shadowradius);
        \shade[outercolor,inner color=innercolor,outer color=outercolor] ($(#1.north west)+(\shadowshift)+(\shadowradius/2,-\shadowradius/2)$) circle (\shadowradius);
        \shade[outercolor,inner color=innercolor,outer color=outercolor] ($(#1.south east)+(\shadowshift)+(-\shadowradius/2,\shadowradius/2)$) circle (\shadowradius);
        \shade[outercolor,inner color=innercolor,outer color=outercolor] ($(#1.north east)+(\shadowshift)+(-\shadowradius/2,-\shadowradius/2)$) circle (\shadowradius);
        \shade[top color=innercolor,bottom color=outercolor] ($(#1.south west)+(\shadowshift)+(\shadowradius/2,-\shadowradius/2)$) rectangle ($(#1.south east)+(\shadowshift)+(-\shadowradius/2,\shadowradius/2)$);
        \shade[left color=innercolor,right color=outercolor] ($(#1.south east)+(\shadowshift)+(-\shadowradius/2,\shadowradius/2)$) rectangle ($(#1.north east)+(\shadowshift)+(\shadowradius/2,-\shadowradius/2)$);
        \shade[bottom color=innercolor,top color=outercolor] ($(#1.north west)+(\shadowshift)+(\shadowradius/2,-\shadowradius/2)$) rectangle ($(#1.north east)+(\shadowshift)+(-\shadowradius/2,\shadowradius/2)$);
        \shade[outercolor,right color=innercolor,left color=outercolor] ($(#1.south west)+(\shadowshift)+(-\shadowradius/2,\shadowradius/2)$) rectangle ($(#1.north west)+(\shadowshift)+(\shadowradius/2,-\shadowradius/2)$);
        \filldraw ($(#1.south west)+(\shadowshift)+(\shadowradius/2,\shadowradius/2)$) rectangle ($(#1.north east)+(\shadowshift)-(\shadowradius/2,\shadowradius/2)$);
    \end{pgfonlayer}
}

% create a shadow layer, so that we don't need to worry about overdrawing 
%other things
\pgfdeclarelayer{shadow} 
\pgfsetlayers{shadow,main}

\newcommand\shadowimage[2][]{%
\begin{tikzpicture}
\node[anchor=south west,inner sep=0] (image) at (0,0) {%
  \includegraphics[#1]{#2}};
\drawshadow{image}
\end{tikzpicture}}
%%%%%%%%%%%%%%%%%%%%%%%%%%%%%%%%%%%%%%%%%%%%%%%%%%%%%%%%%%%%%%%%%%%%%%%%%%%%%%%
%%%%%%%%%%%%%%%%%%%%%%%%%%%%%%%%%%%%%%%%%%%%%%%%%%%%%%%%%%%%%%%%%%%%%%%%%%%%%%%



%%%%%%%%%%%%%%%%%%%%%%%%%%%%%%%%%%%%%%%%%%%%%%%%%%%%%%%%%%%%%%%%%%%%%%%%%%%%%%%
%%%%%%%%%%%%%%%%%%%%%%%% CAPTION SETTINGS %%%%%%%%%%%%%%%%%%%%%%%%%%%%%%%%%%%%%
\addto\captionsenglish{\renewcommand{\figurename}{Fig.}}%CHANGE FIGURE TO FIG.

\captiontitlefont{\captionfont\small}
\captionnamefont{\captionfonttitle\color{specialblue}\bfseries}% MEMOIR MACRO
\renewcommand{\subcapfont}{%
  \captionfont\small}%SUBFIGURE MACRO
\renewcommand{\subcaplabelfont}{\captionfonttitle\color{specialblue}\bfseries}%SUBFIGURE MACRO
\renewcommand{\subfigbottomskip}{0pt}% SUBFIGURE MACRO
\renewcommand{\subfigtopskip}{0pt}% SUBFIGURE MACRO
\captiondelim{.\thinspace}% spaces are not ignored here
\newcommand*{\postcaptionrule}{\color{specialblue}\rule{\textwidth}{1.5pt}}
\postcaption{\postcaptionrule}
%\setlength{\belowcaptionskip}{10pt}
\makeatletter
\newcommand\nocaptioninlist{\renewcommand\ext@figure{ll}%
                                \renewcommand\ext@table{ll}}%Ignore LOF,LOT
\makeatother

%  Side captions
% Define raggedyleft
%http://tex.stackexchange.com/questions/84465/raggedyleft-in-memoir/84470
\makeatletter
\newcommand{\raggedyleft}[1][2em]{%
  \let\\\@centercr
  \memRTLleftskip\z@ \@plus #1\relax
  \memRTLrightskip\z@skip
  \parindent\z@
  \parfillskip\z@skip
}
\makeatother
\renewcommand*{\sidecapstyle}{%
\captiontitlefont{\captionfont\small}
\captionnamefont{\captionfonttitle\color{specialblue}\bfseries}
\ifscapmargleft%
  \captionstyle{\raggedyright[0.5em]}
\else
\captionstyle{\raggedyleft[0.5em]}
\fi
}

\sidecapmargin{outer} % Place in the outer margin
\setsidecappos{b} % Align the caption at the center of the float
\addtolength{\sidecapraise}{-1.1ex}%looks better
% ----------------- %%%%%%%%%%%%%%%%%%%%%%%%%%%%%%%%%%%%%%%%%%%%%%%%%%%%%%%%%%% 
% Raggedsidecaption %%%%%%%%%%%%%%%%%%%%%%%%%%%%%%%%%%%%%%%%%%%%%%%%%%%%%%%%%%%
% ----------------- %%%%%%%%%%%%%%%%%%%%%%%%%%%%%%%%%%%%%%%%%%%%%%%%%%%%%%%%%%%
\newlength{\scapindentleft}%=\textwidth-\maxsidecapfloatwidth
\newlength{\scapindentright}%=0pt
\newlength{\maxsidecapfloatwidth}%      
\newlength{\sidecapintomargin}%=\totalmarginwidth
\newlength{\scapdiffmax}%the difference between maxfloatwidth and the realone
\setlength{\scapdiffmax}{0pt}%otherthan zero when float < maxsidecapfloatwidth

%\setlength{\sidecapsep}{0.04\textwidth} % Distance between float and caption 
\setlength{\maxsidecapfloatwidth}{0.8\textwidth}
\setlength{\scapindentleft}{\textwidth}
\addtolength{\scapindentleft}{-\maxsidecapfloatwidth}
\setlength{\scapindentright}{0pt}%to be modified in environment)
\setlength{\sidecapintomargin}{\totalmarginwidth}
\setlength{\sidecapwidth}{\totaltextwidth}%=totaltextwidth-sidecapsep-maxside..
\addtolength{\sidecapwidth}{-\sidecapsep}
\addtolength{\sidecapwidth}{-\maxsidecapfloatwidth}

% Arguments:
% 1: (Optional) the label for the caption
% 2: The width of the float
% 3: Caption text
% Between begin and end: Everything there is placed in the float.

\NewDocumentEnvironment{raggedsidecap}{m m o m}{%
  \begin{figure}
  \renewcommand{\sidecapfloatwidth}{#2}
  \setlength{\scapdiffmax}{\maxsidecapfloatwidth}
  \addtolength{\scapdiffmax}{-\sidecapfloatwidth}
  
  \addtolength{\sidecapsep}{0.5\scapdiffmax}
  
  \addtolength{\scapindentleft}{0.5\scapdiffmax}
  
  \addtolength{\scapindentright}{0.5\scapdiffmax}
  %\strictpagecheck
  
  \IfNoValueTF{#3}%Check if the optional caption is given.
    {\nocaptioninlist\begin{sidecaption}{#4}[#1]}%no caption in LOF
    {\begin{sidecaption}[#3]{#4}[#1]}%with caption in LOF
  }
  {\end{sidecaption}\end{figure}}
%Exactly the same environment as above but for tables not figures.
\NewDocumentEnvironment{raggedsidecaptab}{m m o m}{%
  \begin{table}
  \renewcommand{\sidecapfloatwidth}{#2}
  \setlength{\scapdiffmax}{\maxsidecapfloatwidth}
  \addtolength{\scapdiffmax}{-\sidecapfloatwidth}
  
  \addtolength{\sidecapsep}{0.5\scapdiffmax}
  
  \addtolength{\scapindentleft}{0.5\scapdiffmax}
  
  \addtolength{\scapindentright}{0.5\scapdiffmax}
  %\strictpagecheck

  \IfNoValueTF{#3}
    {\nocaptioninlist\begin{sidecaption}{#4}[#1]}
    {\begin{sidecaption}[#3]{#4}[#1]}
  }
  {\end{sidecaption}\end{table}}
%  
% Here be dragons. Overwrite of memoir methods. Do not touch  
\makeatletter  
\def\endsidecaption{%
  \m@mscapend@fbox%
  \refstepcounter\@captype
  \m@mscaplabel%
  \m@mscapcheckside%<---- added this
  \begin{lrbox}{\m@mscap@capbox}%
    \begin{minipage}[c]{\sidecapwidth}%
      \sidecapstyle
      \@caption\@captype[\m@mscap@fortoc]{\m@mscap@forcap}
    \end{minipage}%
  \end{lrbox}%
  \m@mscapopboxes}%
\renewcommand*{\m@mscapopboxes}{%
  % Correctly indented from margin
  \ifscapmargleft%
    \hspace{\scapindentleft}%
  \else%.
    \hspace{\scapindentright}%
  \fi%
  %
  \m@mcalcscapraise%
  \usebox{\m@mscap@fbox}%\m@mscapcheckside %<--- removed here
  \ifscapmargleft%
  \rlap{\kern-\m@mscaplkern%
    \raisebox{\m@mscapraise}{\usebox{\m@mscap@capbox}}}%
  \else%
  \rlap{\kern\sidecapsep
    \raisebox{\m@mscapraise}{\usebox{\m@mscap@capbox}}}%
  \fi%
  \gdef\m@mscapthisside{}%
  \@mem@scap@afterhook%
}
\makeatother
%%%%%%%%%%%%%%%%%%%%%%%%%%%%%%%%%%%%%%%%%%%%%%%%%%%%%%%%%%%%%%%%%%%%%%%%%%%%%%%
%%%%%%%%%%%%%%%%%%%%%%%%%%%%%%%%%%%%%%%%%%%%%%%%%%%%%%%%%%%%%%%%%%%%%%%%%%%%%%%



%%%%%%%%%%%%%%%%%%%%%%%%%%%%%%%%%%%%%%%%%%%%%%%%%%%%%%%%%%%%%%%%%%%%%%%%%%%%%%%
%%%%%%%%%%%%%%%%%%%%%%%% Footnote Settings %%%%%%%%%%%%%%%%%%%%%%%%%%%%%%%%%%%%
\footmarkstyle{\textsuperscript{#1}}
\renewcommand*{\foottextfont}{\normalfont}%This is not working properly
%\renewcommand*{\footnoterule}{}
%%%%%%%%%%%%%%%%%%%%%%%%%%%%%%%%%%%%%%%%%%%%%%%%%%%%%%%%%%%%%%%%%%%%%%%%%%%%%%%
%%%%%%%%%%%%%%%%%%%%%%%%%%%%%%%%%%%%%%%%%%%%%%%%%%%%%%%%%%%%%%%%%%%%%%%%%%%%%%%



%%%%%%%%%%%%%%%%%%%%%%%%%%%%%%%%%%%%%%%%%%%%%%%%%%%%%%%%%%%%%%%%%%%%%%%%%%%%%%%
%%%%%%%%%%%%%%%%%%%%%%%% MISC COMMANDS FOR THE TEXT     %%%%%%%%%%%%%%%%%%%%%%%
\externaldocument[A:]{chap1}% Cross referencing first chapter
\externaldocument[B:]{chap2}% Cross referencing second chapter
\externaldocument[C:]{appendices}% Cross referencing appendix chapter
\newcommand*{\um}[1]{\SI{#1}{\micro\metre}}
\newcommand*{\mm}[1]{\SI{#1}{\milli\metre}}
\newcommand*{\kpa}[1]{\SI{#1}{\kilo\pascal}}
\newcommand*{\percents}[1]{\SI{#1}{\percent}}
\newcommand*{\temperature}[1]{\SI{#1}{\degreeCelsius}}
\newcommand*{\volts}[1]{\SI{#1}{\volt}}
\strictpagecheck
%%%%%%%%%%%%%%%%%%%%%%%%%%%%%%%%%%%%%%%%%%%%%%%%%%%%%%%%%%%%%%%%%%%%%%%%%%%%%%%
%%%%%%%%%%%%%%%%%%%%%%%%%%%%%%%%%%%%%%%%%%%%%%%%%%%%%%%%%%%%%%%%%%%%%%%%%%%%%%%



%%%%%%%%%%%%%%%%%%%%%%%%%%%%%%%%%%%%%%%%%%%%%%%%%%%%%%%%%%%%%%%%%%%%%%%%%%%%%%%
%%%%%%%%%%%%%%%%%%%%%%%% FONTSPEC PROBLEM WITH FOOTNOTE %%%%%%%%%%%%%%%%%%%%%%%
% http://tex.stackexchange.com/questions/72803/scaling-problem-with-fontspec-pa
%ckage-and-footnote
\makeatletter
\renewcommand*{\@makefnmark}{%
  \hbox{%
    \@textsuperscript{%
      % \normalfont % removed
      \selectfont % added
      \@thefnmark
    }%
  }%  
}   
\makeatother
%%%%%%%%%%%%%%%%%%%%%%%%%%%%%%%%%%%%%%%%%%%%%%%%%%%%%%%%%%%%%%%%%%%%%%%%%%%%%%%
%%%%%%%%%%%%%%%%%%%%%%%%%%%%%%%%%%%%%%%%%%%%%%%%%%%%%%%%%%%%%%%%%%%%%%%%%%%%%%%



%%%%%%%%%%%%%%%%%%%%%%%%%%%%%%%%%%%%%%%%%%%%%%%%%%%%%%%%%%%%%%%%%%%%%%%%%%%%%%%
%%%%%%%%%%%%%%%%%%%%%%%% STYLE THE REFERENCES BASED ON IEEE %%%%%%%%%%%%%%%%%%%
% http://tex.stackexchange.com/questions/65379/reset-the-counter-of-references-
%in-each-chapter/72873#72873
\addto\captionsenglish{\renewcommand{\refname}{References}}
\ExecuteBibliographyOptions{maxnames=5,minnames=1,minbibnames=5,maxbibnames=5,%
maxcitenames=1,refsection=chapter,firstinits=true,backref=true,backrefstyle=three,%
block=space,sortcites=true}%dashed=false}% will be added when support comes
\renewbibmacro*{bbx:savehash}{} % Remove annoying dash
\DeclareFieldFormat{labelnumberwidth}{%
\hspace{2.5pt}\textit{\bfseries\large#1}\adddot\space}
\defbibenvironment{bibliography}
  {\list
    {\printfield[labelnumberwidth]{labelnumber}}
    {\setlength{\itemindent}{0pt}%
    \setlength{\labelwidth}{\labelnumberwidth}%
    \settowidth{\labelsep}{\space}%
    \setlength{\leftmargin}{\labelnumberwidth}
    \setlength{\itemsep}{\bibitemsep}%
    \setlength{\parsep}{\bibparsep}}}%
    {\endlist}
   {\item}

\newcommand{\printreferences}{%
%\clearpage%
%\begin{adjmulticols}{1}{}{-\totalmarginwidth}%
\printbibliography[section=\therefsection,heading=myref]%
%\end{adjmulticols}
}
\renewcommand{\bibfont}{\small}
\renewcommand*{\bibsetup}{%
  \hyphenpenalty= 50\relax
  \tolerance=800\relax
  \interlinepenalty=5000\relax
  \widowpenalty=10000\relax
  \clubpenalty=10000\relax
  \raggedbottom
  \SingleSpacing %Added by me
  \frenchspacing 
  \biburlsetup}
\DefineBibliographyStrings{english}{% redefine 'cit. on' to 'cited on'
  backrefpage = {cited on p\adddot},
  backrefpages= {cited on pp\adddot},
}
\defbibheading{myref}[\refname]{%
  \section{#1}}
%\setcounter{biburlnumpenalty}{100}
%\setcounter{biburlucpenalty}{100}
%\setcounter{biburllcpenalty}{100}
%\setcounter{abbrvpenalty}{100}
%\setcounter{highnamepenalty}{1000}
%%%%% Customization of ieee style %%%%%%%%%%%%%
\AtEveryBibitem{% remove unwanted fields
  \clearfield{issn}
  \clearfield{eprint}
  \clearfield{isbn}
  \clearfield{pmid}
  \clearfield{month}
}

\DeclareFieldFormat{url}{%
  \mkbibbrackets{\href{#1}{Online}}
}

\DeclareFieldFormat{doi}{%
  \iffieldundef{url}
    {\mkbibbrackets{\href{http://dx.doi.org/#1}{Online}}}% url is not defined
    {}%skip doi since url is already available
}
%%%%%%%%%%%%%%%%%%%%%%%%%%%%%%%%%%%%%%%%%%%%%%%%%%%%%%%%%%%%%%%%%%%%%%%%%%%%%%%
%%%%%%%%%%%%%%%%%%%%%%%%%%%%%%%%%%%%%%%%%%%%%%%%%%%%%%%%%%%%%%%%%%%%%%%%%%%%%%%



%%%%%%%%%%%%%%%%%%%%%%%%%%%%%%%%%%%%%%%%%%%%%%%%%%%%%%%%%%%%%%%%%%%%%%%%%%%%%%%
%%%%%%%%%%%%%%%%%%%%%%%% JUST FOR DEBUGGING             %%%%%%%%%%%%%%%%%%%%%%%
\usepackage{blindtext}
%\usepackage{showlabels}
%\usepackage{lineno}
%\pagewiselinenumbers
%%%%%%%%%%%%%%%%%%%%%%%%%%%%%%%%%%%%%%%%%%%%%%%%%%%%%%%%%%%%%%%%%%%%%%%%%%%%%%%
%%%%%%%%%%%%%%%%%%%%%%%%%%%%%%%%%%%%%%%%%%%%%%%%%%%%%%%%%%%%%%%%%%%%%%%%%%%%%%%
